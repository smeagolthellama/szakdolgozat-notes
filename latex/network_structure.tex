\chapter{A h\'al\'ozat strukt\'ur\'aja}

A protokoll központi eleme a hálózat, amely egy fa típusú konfigurációt
vesz majd fel. A gyökérben található az adatforrás, amíg a gyerekek a
hallgató közönség.

A fa szerkezetére két lehetőség van, amik között még a tervezés folyamán
döntést kellene hozni: legyen bináris vagy sem.

Ha a fa bináris, akkor lehet garantálni ennek sok kellemes tulajdonságát,
amelyek kihasználásával a hálózat elrendezését és a csatlakozási
folyamatot nagy mértékben lehetne egyszerűsíteni, és ezért gyorsítani is.

Ha a fa nem bináris, akkor a felek egyénileg el tudják dönteni mennyire
képesek, és így sokkal kevesebb lenne a fa mélysége, vagyis az átlagos
késés sokat csökkenne, de ekkor több adatot kell több helyre elküldeni,
ami a protokollt is, a program logikáját is bonyolítaná.

\begin{figure}
%

%

%

%
%
%

% Start of code
% \begin{tikzpicture}[anchor=mid,>=latex',line join=bevel,]
\begin{tikzpicture}[>=latex',line join=bevel,]
  \pgfsetlinewidth{1bp}
%%
\pgfsetcolor{black}
  % Edge: server -> client1
  \draw [->] (62.154bp,52.665bp) .. controls (72.239bp,55.363bp) and (83.764bp,58.445bp)  .. (104.3bp,63.939bp);
  % Edge: server -> client2
  \draw [->] (62.154bp,37.335bp) .. controls (72.239bp,34.637bp) and (83.764bp,31.555bp)  .. (104.3bp,26.061bp);
  % Edge: client1 -> client3
  \draw [->] (170.13bp,72.0bp) .. controls (178.22bp,72.0bp) and (186.97bp,72.0bp)  .. (205.57bp,72.0bp);
  % Edge: client2 -> client4
  \draw [->] (170.13bp,18.0bp) .. controls (178.22bp,18.0bp) and (186.97bp,18.0bp)  .. (205.57bp,18.0bp);
  % Node: server
\begin{scope}
  \definecolor{strokecol}{rgb}{0.0,0.0,0.0};
  \pgfsetstrokecolor{strokecol}
  \draw (32.5bp,45.0bp) ellipse (32.49bp and 18.0bp);
  \draw (32.497bp,45.0bp) node {server};
\end{scope}
  % Node: client1
\begin{scope}
  \definecolor{strokecol}{rgb}{0.0,0.0,0.0};
  \pgfsetstrokecolor{strokecol}
  \draw (135.44bp,72.0bp) ellipse (34.39bp and 18.0bp);
  \draw (135.44bp,72.0bp) node {client1};
\end{scope}
  % Node: client2
\begin{scope}
  \definecolor{strokecol}{rgb}{0.0,0.0,0.0};
  \pgfsetstrokecolor{strokecol}
  \draw (135.44bp,18.0bp) ellipse (34.39bp and 18.0bp);
  \draw (135.44bp,18.0bp) node {client2};
\end{scope}
  % Node: client3
\begin{scope}
  \definecolor{strokecol}{rgb}{0.0,0.0,0.0};
  \pgfsetstrokecolor{strokecol}
  \draw (240.34bp,72.0bp) ellipse (34.39bp and 18.0bp);
  \draw (240.34bp,72.0bp) node {client3};
\end{scope}
  % Node: client4
\begin{scope}
  \definecolor{strokecol}{rgb}{0.0,0.0,0.0};
  \pgfsetstrokecolor{strokecol}
  \draw (240.34bp,18.0bp) ellipse (34.39bp and 18.0bp);
  \draw (240.34bp,18.0bp) node {client4};
\end{scope}
%
\end{tikzpicture}
% End of code

%
%




\caption{hogyan nézhetne ki a hálózat négy klienssel}
\end{figure}

A teljes fára nézett átlag késés minimalizálása érdekében, egy
meritokratikus rangsorolás szerint a jobb hálózati kapcsolattal
rendelkező kliensek közelebb kerülhetnek a gyökérhez, és így gyorsabban
továbbadhatják a kapott adatokat az ők klienseiknek.
