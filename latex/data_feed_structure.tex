\chapter{Az adatfolyam struktúrája}

A hálózat tagjai között a közvetített adatok bizonyos extra
metainformációval lesz társítva. Ebben a fejezetben le lesznek ezek írva,
azzal együtt, hogy miért vannak ott.

\section{Aláírás}

Mivel elméletileg a közvetített adat megszakítás nélküli, de a hálózaton
csomagkapcsolás van, az adatforrás által adott adatokat először bizonyos,
az adat típusa ismeretében megválasztott méretű csomagokra lesz választva.
Ezután, az adatok hitelességének garanciája érdekében, a szerver egy
digitális aláírást számít az adott csomagra, és esetleg egy hash-et is,
ha szükséges. A szerver publikus kulcsa már csatlakozás előtt ismert kell
legyen a kliensek számára, ezért ezt nem közvetítjük, de a csomagok
méretét már igen.

Megeshet, hogy az adatsebesség miatt nem praktikus minden csomagot
aláírni, ezért lehetséges az, hogy csak minden k-adik csomaggal társul
aláírás, de ez valahogy mind a k csomagra érvényes kell legyen.

\section{csomagtípus-azonosító}

Mivel több csomag típus is fog létezni, lesz egy mező a csomagokban,
amely azonosítja, hogy milyen csomag következik: Aláírt, Aláírás-mentes,
rendezési, vagy hálózatfenntartási.

\section{sorrendjelölő}

Mivel a protokoll UDP alapú lesz, nem létezik garancia arra, hogy
sorrendben érkeztek-e az csomagok. Egyes esetekben, fontos lehet az
adatok sorrendbe-helyezése, és erre a célra szolgál a sorrendjelző. 
Ez a mező nem fogja számon tartani azt, hogy eddig hány csomagot küldött
a szerver, mivel hosszas közvetítés után a mező mérete impraktikusan nagy
lenne. Ezért nagyjából csak egy bájt méret lesz erre a szerepre
félretéve.
