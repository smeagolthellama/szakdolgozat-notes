\chapter{Bevezet\H{o}}\label{ch:BEV}

\section{Motiváció}\label{sec:BEV:mot}

A peer-to-peer információ-megosztó protokollok,
mint például a BitTorrent, IPFS, és mások az
adatok integritását helyezik el\H{o}térbe. Ez
sokszor igenis fontos, de léteznek esetek,
amikor más szempont fontosabb lehet. Ezeknek az
eseteknek egy nagy részében, az adatok gyors
megérkezése fontosabb mint az adatok egyben
maradása. Ez a kett\H{o}s adat-továbbítási
prioritás a kliens-szerver modellnél nagyon
tisztán látható, abban, hogy a két alap
protokoll (a TCP és az UDP) pont ezt a két
igényt fedezi.

Ezeknek a protokolloknak gyakran reklámozott
része, hogy az adatbiztonság mellett a megosztó
fél identitása is védett. Habár ez sok esetben
kecsegtet\H{o}, bizonyos esetekben fontos lehet az,
hogy bizonyíthassa valaki, hogy honnan szedte az
adatait.

A fentiek ismeretében, ebben a dolgozatban le
lesz írva egy protokoll vázlata, ami a
következ\H{o}ket teljesíti:
\begin{itemize}
\item Egyetlen, hitelességgel rendelkez\H{o}
forrásból eljuttat adatokat gyorsan, számos
érdekelt félhez.
\item Semelyik fél se legyen túlságosan
leterhelve
\item Bármelyik fél tudja bizonyítani, hogy az
adat amit kapott a forrástól jön, és hiteles
\item Az ellenfeles viselkedés\H{u} feleket fel
tudja ismerni, azonosítani, és minimalizálni
befolyásukat
\item hatékonyan létrehoz egy hálózatot az
érdekelt felek között
\item tudjon nagy mennyiségű időtől függő adatokat közvetíteni
\end{itemize}


\section{lehetséges felhasználási területek}

Olyan adatokról volna szó, amiknek csak a "jelenlegi" értéke számít. Nem
probléma, ha az elejéről akármennyi elvesztődik, sőt, a közepéről sem.

Példák felhasználási területekre lehetnének a műholdak szenzoradatai
nyilvános hozzáférési módszere, vagy akár más közérdekű tudományos
kísérlet mérései, vagy a szórakoztatás terén az internetes videós
élő közvetítéseket is lehetne ezen a hálózaton hordozni.
