\chapter{Bevezető}\label{ch:BEV}

\section{Motiváció}\label{sec:BEV:mot}

A peer-to-peer információ-megosztó protokollok,
mint például a BitTorrent, IPFS, és mások az
adatok integritását helyezik előtérbe. Ez
sokszor igenis fontos, de léteznek esetek,
amikor más szempont fontosabb lehet. Ezeknek az
eseteknek egy nagy részében, az adatok gyors
megérkezése fontosabb mint az adatok egyben
maradása. Ez a kettős adat-továbbítási
prioritás a kliens-szerver modellnél nagyon
tisztán látható, abban, hogy a két alap
protokoll (a TCP és az UDP) pont ezt a két
igényt fedezi.

Ezeknek a protokolloknak gyakran reklámozott
része, hogy az adatbiztonság mellett a megosztó
fél identitása is védett. Habár ez sok esetben
kecsegtető, bizonyos esetekben fontos lehet az,
hogy bizonyíthassa valaki, hogy honnan szedte az
adatait.

A fentiek ismeretében, ebben a dolgozatban le
lesz írva egy protokoll vázlata, ami a
következőket teljesíti:
