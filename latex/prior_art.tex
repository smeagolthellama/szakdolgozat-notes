\chapter{L\'{e}tez\H{o} komponensek}

\section{Létező peer-to-peer adat-megosztó megoldások}

A létező, népszerű, p2p adat-megosztó megoldások az adat fajtájától
függően két nagy kategóriába sorolhatóak (amiknek létezik nem üres metszete):

1) Állománymegosztó protokollok
(mint például a már említett IPFS \citep{Benet} vagy BitTorrent, vagy akár a croc nevű
nyitott forráskódú applikáció, vagy a Gnutella protokoll, ami a Morpheus
és a LimeWire applikációk alapját szolgálta) 2) Instans-üzenet-küldő
programok (például a tox protokoll \citep{enwiki:1159250538}, vagy az IRC).

Habár bármilyen adatot amit a számítógép tud kezelni le lehet menteni egy
állományba, és bizonyos lépéseken keresztül szöveges üzenetekbe is át
lehet alakítani (például base64 encoding), ezek a protokollok egyszerűen
nem alkalmasak arra, hogy egy állandóan frissülő adatforrást
közvetítsenek.

\section{létező peer-to-peer "streaming" megoldások}

Apache kafka: igazából kliens-szerver megoldás, és az adatok
feldolgozásával/kategorizálásával foglalkozik, nem a kezdeti
közvetítéssel.

Strivecast\citep{st}: explicit a videó közvetítését szolgálja, a
WebRTC technológia segítségével. Szükséges szerver, hogy elintézze a
kezdeti csatlakozást. A peer5 ugyancsak. Ezen felül, ez a szoftver a
kétirányú kommunikációt helyezi előtérbe, ami sokszor nem fontos.

Seedess\citep{seed}: Bittorent alapú, és reklámjai szerint csak csökkenti a szerverek
terhelését, és ezt is csak 70\%-osan. A példa amit adnak egy
videóállomány, vagyis nem "élő adás".
